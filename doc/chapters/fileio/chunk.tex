The chunk file contains serialized info for the chunks. The data should be height data, zone list, ogd list, override list, inhabited zone, inhabited override.

\subsection{Header}
The metablock custom data is 4 bytes for x axis index, 4 for y.

\subsection{Chunk Depth}
    Chunks hold onto data related to intradimensionality. For this, a u16 denotes the dimensional layers within the chunk. Each layer will be listed subsequently by a u8 mask to denote what entries exist within that layer.
    
    \begin{itemize}
        \item height data     0000 0001
        \item volume data     0000 0010
        \item zone data       0000 1000
        \item AGD list        0010 0000
        \item override tables 1000 0000
    \end{itemize}

\subsection{Chunk Archive}

    Each record is prefaced with a u16 for current dimensional depth layer, the bitmask value associated to the record, and a u32 for the record length. These values repeat for error checking purposes to prevent data mangling/corruption on malformed files.

\subsection{Height Data}
    Height data is the primary mesh data for generating terrain. It contains a flag to indicate whether the terrain is upward or downward facing,
    Height data is represented as a unsigned 16-bit integer, every two units(meters). It is multiplied by 0.1 for granularity, and added to a height offset. The offset is an unsigned 8-bit integer, multiplied by 1024.0. The storage for height data is as follows:

\begin{itemize}
    \item start with last height at 0, starting height is expected above that
    \item if current height is less than 10.0 +- last height save current height as u8. (delta - 100)
    \item else if it is greater than, save the whole u16 with leading with max u8 (255)
    \item set current height as last height and continue
\end{itemize}

\subsection{Volumetric Data}

Volumetric data is a pair of u16s, separated out by a pai value of number of skips before next volume entry, and for how many entries follow, at which point it repeats another pair until the chunk is complete. 

\subsection{Zone Data}
    Zones are stored in a linear fashion with a value pair for integral zone type and number of steps to next zone. The zone type is a u32, and the step count is a u16. Steps are odd counted, iterating over every second unit.

\subsection{Override Table}
    Override Tables are a delta applied to the entries in a chunk as deltas to heights, AGDs, zones, time, and other necessary data. Override tables cannot swap the index of the chunk. Override tables are processed linearly, overriding prior entries.
    
    AGD overrides have an unpacked AGD with the AID (Asset IDentification number) to override existing elements.


